
\section{Research Objectives and Questions}
\label{sec:objectives}

% This section addresses ILO2: Problem formulation and research questions
\subsection{Overall Aim}

The overall aim of this research is to investigate and conceptually design how emerging agent communication protocols, specifically Agent Communication Protocol (ACP) and Agent-to-Agent Protocol (A2A), can be applied and adapted to enable secure, scalable, and interoperable communication for decentralized predictive maintenance coordination among diverse Distributed Energy Resources (DERs) owned and operated by different entities, and to develop a quantitative framework for evaluating their performance in this context.

\subsection{Specific Objectives}

The specific objectives of this research are:
\begin{enumerate}
    \item \textbf{To analyze} the communication requirements for decentralized predictive maintenance coordination among diverse distributed energy resources (DERs) owned and operated by different entities.
    \item \textbf{To investigate} the key features and capabilities of Agent Communication Protocol (ACP) and Agent-to-Agent Protocol (A2A) relevant to fulfilling these communication requirements.
    \item \textbf{To design} a conceptual framework illustrating how ACP and/or A2A can be applied to facilitate the defined communication patterns for DER predictive maintenance.
    \item \textbf{To specify} core messaging patterns and information exchange requirements using ACP and/or A2A primitives and concepts for DER health data exchange and maintenance task initiation.
    \item \textbf{To develop} a quantitative evaluation framework for assessing the performance of ACP and A2A in DER predictive maintenance scenarios, identifying relevant metrics and evaluation criteria through literature review.
\end{enumerate}

\subsection{Research Questions}

\subsubsection{Primary Research Question}

How can Agent Communication Protocol (ACP) and Agent-to-Agent Protocol (A2A) be applied and adapted to enable secure, scalable, and interoperable communication for decentralized predictive maintenance coordination among diverse, multi-owner Distributed Energy Resources (DERs), and what quantitative framework can be developed to evaluate their performance in this context?

\subsubsection{Sub-Questions}

Based on the primary research question and specific objectives, the following sub-questions will guide the research:

\begin{enumerate}
    \item \textbf{Sub-Question 1:} What are the key communication requirements for enabling decentralized predictive maintenance coordination among diverse Distributed Energy Resources (DERs) owned and operated by different entities?
    
    \item \textbf{Sub-Question 2:} How can the features and messaging patterns of Agent Communication Protocol (ACP) and Agent-to-Agent Protocol (A2A) be applied and potentially adapted to address these requirements, ensuring secure, scalable, and interoperable communication for DER predictive maintenance coordination?
    
    \item \textbf{Sub-Question 3:} What metrics and evaluation criteria should be used to assess the performance of ACP and A2A in the context of DER predictive maintenance coordination, and how can these be identified through literature review?
\end{enumerate}


\section{Scope and Limitations}
\label{sec:scope}

\subsection{Scope}

The research is bounded by the following scope parameters:

\subsubsection{Protocol Focus}
The analysis and conceptual design will be strictly limited to the application and adaptation of the Agent Communication Protocol (ACP) and Agent-to-Agent Protocol (A2A). Other agent protocols or general communication standards will not be explored in detail.

\subsubsection{Use Case Specificity}
The research focuses solely on the predictive maintenance coordination use case for DERs. Applications of ACP/A2A in other DER functions (e.g., energy trading, grid stability services) are outside the scope.

\subsubsection{Quantitative Framework Development}
The research will develop a structured framework for measuring and comparing protocol performance through:
\begin{itemize}
    \item Literature review to identify relevant performance metrics and evaluation methodologies
    \item Development of benchmark scenarios based on documented specifications
    \item Establishment of evaluation criteria through analysis of protocol features
    \item Design of comparison methodology for systematic assessment
    \item Limited validation through simplified test cases
\end{itemize}

\subsubsection{Analysis Approach}
The evaluation of protocol suitability will be based on:
\begin{itemize}
    \item Literature review and protocol specifications
    \item Theoretical performance analysis using documented metrics
    \item Scalability assessment based on protocol architecture
    \item Security and reliability evaluation using published data
    \item Results from limited framework validation
\end{itemize}

\subsection{Limitations}

The research acknowledges the following key limitations:

\subsubsection{Time and Resource Constraints}
The research is constrained by the typical timeframe of a Master's thesis (approximately 20 weeks), which necessitates a focused scope and limits the depth of empirical work that can be undertaken.

\subsubsection{Limited Empirical Validation}
While the research develops a quantitative evaluation framework, comprehensive empirical validation is limited to simplified test cases due to time constraints. The research relies primarily on theoretical analysis and literature review.

\subsubsection{Protocol Scope}
By focusing exclusively on ACP and A2A, the research may not capture insights from other potentially relevant agent communication protocols. However, this focused approach allows for deeper analysis of the selected protocols.

\subsubsection{DER Type Generalization}
The research uses a generalized approach to DER characteristics rather than detailed analysis of specific DER types (e.g., solar, wind, battery storage). This ensures broader applicability while potentially missing type-specific requirements.

\subsubsection{Security Analysis Depth}
Security considerations are addressed at a conceptual level without detailed cryptographic protocol design or implementation, which would require expertise and time beyond the thesis scope.

\subsubsection{Data Accessibility}
Obtaining real-world, high-resolution predictive maintenance data from diverse, multi-owner DER fleets is challenging due to privacy concerns and proprietary systems. The research design acknowledges this and relies on publicly available information and generalized requirements.

\subsection{Mitigation Strategies}

To address these limitations, the research employs the following strategies:
\begin{itemize}
    \item Focus on high-impact aspects of the research that provide maximum value
    \item Leverage existing literature and established frameworks to support theoretical analysis
    \item Clearly document all assumptions and limitations
    \item Provide comprehensive recommendations for future empirical validation
    \item Design the framework to be extensible for future research
\end{itemize}
