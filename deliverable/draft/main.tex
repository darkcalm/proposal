\documentclass[12pt,a4paper]{article}

\usepackage{geometry}
  \usepackage[T1]{fontenc}
\usepackage{mathpazo} % Palatino font
\usepackage{microtype} % Better typography
\usepackage{textcomp} % For proper dash support
\usepackage{amsfonts}
\usepackage{amsmath}
\usepackage{graphicx}
\usepackage{float}
\usepackage{hyperref}
\hypersetup{
    colorlinks=true,
    linkcolor=black,
    filecolor=magenta,
    urlcolor=blue,
    citecolor=black,
}

% Typography settings
\linespread{1.05} % Slightly increased line spacing
\setlength{\parindent}{1.5em}
\setlength{\parskip}{0.5em}
\setlength{\textwidth}{6.5in}
\setlength{\textheight}{9in}
\setlength{\oddsidemargin}{0in}
\setlength{\evensidemargin}{0in}
\setlength{\topmargin}{0in}
\setlength{\headheight}{15pt}
\setlength{\headsep}{40pt}
\setlength{\footskip}{30pt}

% Additional typography packages
\usepackage{ragged2e}
\usepackage{booktabs}
\usepackage[inline]{enumitem}
\usepackage{multicol}
\usepackage{multirow}
\usepackage{tikz}
\usepackage{xcolor}
\usepackage{longtable}
\usepackage{appendix}
\usepackage{tabularx}
\usepackage{threeparttable}
\usepackage{pdflscape}
\usepackage{array}
\usepackage{pgfgantt}
\usepackage{setspace}
\usepackage{fancyhdr}
\usepackage[sorting=none,backend=biber]{biblatex}

\addbibresource{main.bib}

\pagestyle{fancy}
\fancyhf{}
\fancyhead[L]{Course - Methods of Research in Sustainable Energy}
\fancyhead[R]{Spring 2025}
\fancyfoot[C]{\thepage}

%%%%%%%%%%%%%%%%%%%%%%%%%%%%%%%%%%%%%%%%%%%%%%%%%%
% ENTER GROUP AND PROJECT INFORMATION
%%%%%%%%%%%%%%%%%%%%%%%%%%%%%%%%%%%%%%%%%%%%%%%%%%

\newcommand{\grouptitle}{Group [X] - Research Proposal}
\newcommand{\studentone}{Student Name 1}
\newcommand{\studenttwo}{Student Name 2}
\newcommand{\studentthree}{Student Name 3}
\newcommand{\projecttitle}{[Enter Your Research Title Here]}
\newcommand{\submissiondate}{May 28, 2025}
\newcommand{\emdash}{\textemdash}

%%%%%%%%%%%%%%%%%%%%%%%%%%%%%%%%%%%%%%%%%%%%%%%%%%

\begin{document}

% Title Page
\begin{titlepage}
\begin{center}
{\Huge{Research Proposal}} \\
\vspace{5mm}
{\Large{Methods of Research}} \\

\vspace{10mm}

{\huge{\textbf{\projecttitle}}} \\

\vspace{15mm}

\hrule
\vspace{3mm}
\begin{tabular}{ll}
\textbf{Group Members:} & {\studentone} \\
& {\studenttwo} \\
& {\studentthree} \\
\\
\textbf{Submission Date:} & {\submissiondate} \\
\textbf{Word Count:} & [Approximately 2000 words] \\
\end{tabular}
\vspace{3mm}
\hrule

\vspace{15mm}

\textbf{Abstract} \\
\vspace{2mm}
\begin{minipage}{0.8\textwidth}
[Provide a concise abstract (150-200 words) summarizing your research proposal, including the research problem, methodology, and expected contributions to sustainable energy research.]
\end{minipage}

\end{center}
\end{titlepage}

\tableofcontents
\newpage

% Research Proposal: [Your Proposed Title Here - e.g., Enhancing Human-Autonomy Teaming in Distributed Energy Resource Management through AI-driven Agent Communication Protocols]

\textbf{Student:} [Your Name/Group Names]  
\textbf{Supervisor:} [Supervisor's Name]  
\textbf{Date:} 2025-01-21

\section{Abstract (Target: ~150-200 words of 2500 total)}

\emph{(This section needs to be newly written, summarizing the entire proposal. It should briefly introduce the problem, the research gap, the proposed methodology, and expected outcomes. Based on the examiner's feedback, it should be concise and clear.)}

\textbf{Keywords:} Human-Autonomy Teaming, Distributed Energy Resources, AI Agents, Communication Protocols, Multi-Agent Systems, Explainable AI, Power Systems Management.

\section{Table of Contents}


\begin{enumerate}
\item  Introduction and Background
    1.1. Problem Context and Significance
    1.2. Systematic Literature Review Synthesis
    1.3. Research Gap Identification
\item  Research Objectives and Questions
    2.1. Overall Aim
    2.2. Specific Objectives
    2.3. Research Questions
\item  Scope and Limitations
\item  Theoretical Framework
    4.1. Key Concepts
    4.2. Conceptual Model
\item  Research Methodology
    5.1. Selected Methodology Justification
    5.2. Alternative Methodologies Considered
    5.3. Data Collection and Analysis Approach
    5.4. Workflow
\item  Ethics and Sustainability
    6.1. Ethical Considerations
    6.2. Sustainability Dimensions and SDG Alignment
    6.3. Mitigation Strategies
\item  Risk Assessment and Timeline
    7.1. Major Risk Categories and Mitigation
    7.2. Implementation Timeline Overview
    7.3. Resource Requirements
\item  Expected Results and Contributions
\item  References
\item Appendices (Optional)
    10.1. [Example: Detailed Methodology Comparison Matrix]
    10.2. [Example: Full Risk Register]
\end{enumerate}

---

\section{1. Introduction and Background (Target: 630 words)}

\emph{(Primary Sources: \texttt{docs/4.3.7-gap-statement-summary.md}, \texttt{docs/4.2.1.1-patterns-trends-analysis.md}, \texttt{docs/3.1.2-research-direction-selection.md})}
\emph{(Supporting Sources: \texttt{docs/4.3.6-practical-needs-summary.md}, \texttt{docs/4.1.8.6-elicit-results-processing-summary.md})}
\emph{(Key References: \texttt{sources/4.1.1-elicit-results/}, \texttt{sources/4.1.8-elicit-results/})}

\subsection{1.1. Problem Context and Significance (Approx. 180 words)}
The increasing integration of Distributed Energy Resources (DERs) presents significant challenges for traditional power grid management, requiring more agile, responsive, and intelligent coordination mechanisms [CitationNeeded\emph{DERChallengesOverview]. Effective human-autonomy teaming (HAT) is crucial in this evolving landscape, particularly in ensuring that human operators can effectively manage and supervise complex networks of DERs and autonomous agents [CitationNeeded}HATinDER]. This research addresses the critical need for improved communication and coordination protocols between human operators and AI agents within DER management systems, aiming to enhance system stability, efficiency, and operator trust. The significance lies in developing frameworks that support seamless collaboration, mitigate communication gaps, and leverage the strengths of both human expertise and AI capabilities \cite{CitationNeeded_AIinDERManagement}.

\subsection{1.2. Systematic Literature Review Synthesis (Approx. 250 words)}
A systematic review of literature, drawing from sources such as Elicit.com and Semantic Scholar [CitationNeeded\emph{ElicitSearchStrategy, CitationNeeded}SemanticScholarMethodology], reveals several key patterns and trends. Current research highlights advancements in AI for energy systems, multi-agent systems (MAS) for DER control, and human factors in automation [ReferTo\emph{docs/4.2.1.1-patterns-trends-analysis.md]. While many studies explore either the technical aspects of AI agent control or the human interface components, there is a discernible gap in addressing the socio-technical intricacies of communication protocols specifically designed for HAT in dynamic DER environments [CitationNeeded}GapInHATCommunication]. Key themes emerging include the need for explainable AI (XAI) to foster trust, adaptive interfaces for varying operator cognitive loads, and robust protocols that can handle uncertainty and ensure shared situational awareness [ReferTo\emph{docs/4.2.1.1-patterns-trends-analysis.md, CitationNeeded}XAIforTrust]. Industry analyses also point to practical needs for such systems to improve operational efficiency and safety [ReferTo_docs/4.3.6-practical-needs-summary.md].

\subsection{1.3. Research Gap Identification (Approx. 200 words)}
The primary research gap, identified through comprehensive analysis including \texttt{docs/4.3.7-gap-statement-summary.md}, is the lack of a dedicated, empirically validated Agent Communication Protocol (ACP) framework that specifically supports human-autonomy teaming in the context of predictive maintenance and operational decision-making for DERs. While existing ACPs focus on agent-to-agent communication, and HCI research addresses human-system interaction, the nuanced requirements of dynamic, bi-directional communication between human operators and intelligent DER agents—particularly for tasks like anomaly detection, collaborative diagnostics, and intervention planning—remain underexplored [ReferTo_docs/4.3.7-gap-statement-summary.md]. This gap manifests in suboptimal human oversight, potential for miscommunication, and reduced adoption of advanced AI tools in critical DER operations. This research aims to bridge this gap by developing and evaluating a novel ACP tailored to these specific human-agent collaborative needs.

\section{2. Research Objectives and Questions (Target: 300 words)}

\emph{(Primary Sources: \texttt{docs/4.2.4.1-research-questions-refined.md}, \texttt{docs/3.1.2-research-direction-selection.md})}
\emph{(Supporting Sources: \texttt{docs/4.2.4.2-hypotheses-refined.md})}

\subsection{2.1. Overall Aim (Approx. 50 words)}
The overall aim of this research is to design, develop, and evaluate an AI-driven Agent Communication Protocol (ACP) framework that enhances human-autonomy teaming effectiveness in the management and predictive maintenance of Distributed Energy Resources.

\subsection{2.2. Specific Objectives (Approx. 100 words)}

\begin{enumerate}
\item  To identify the key communication requirements and challenges for effective human-AI agent collaboration in DER management contexts, drawing from \texttt{docs/3.1.2-research-direction-selection.md}.
\item  To design a novel ACP incorporating principles of explainability, context-awareness, and trust-building for human-agent interaction.
\item  To develop a prototype system implementing the proposed ACP within a simulated DER operational environment.
\item  To evaluate the impact of the proposed ACP on team performance, operator workload, situational awareness, and trust, as outlined in \texttt{docs/4.2.4.2-hypotheses-refined.md}.
\end{enumerate}

\subsection{2.3. Research Questions (Approx. 150 words)}
\emph{(Based on \texttt{docs/4.2.4.1-research-questions-refined.md})}

\textbf{Primary Research Question:}

\begin{itemize}
\item   RQ1: How can a novel Agent Communication Protocol, specifically designed for human-autonomy teaming, improve the effectiveness, efficiency, and trustworthiness of collaboration between human operators and AI agents in managing Distributed Energy Resources?
\end{enumerate}

\textbf{Sub-Questions:}

\begin{itemize}
\item   SQ1.1: What are the essential information elements and interaction patterns required for effective communication between human operators and AI agents in DER predictive maintenance and operational tasks?
\item   SQ1.2: How can principles of eXplainable AI (XAI) be embedded within the ACP to enhance operator understanding and trust in AI-driven recommendations and actions?
\item   SQ1.3: To what extent does the proposed ACP improve shared situational awareness and reduce operator cognitive load compared to existing communication paradigms in simulated DER environments?
\item   SQ1.4: What are the key design considerations for an ACP that effectively supports adaptive human-agent interaction in dynamic and uncertain DER operational scenarios?
\end{enumerate}

\section{3. Scope and Limitations (Target: 200 words)}

\emph{(Newly written, drawing from understanding of overall project, but referencing \texttt{docs/3.5.1-Define-boundaries.md} implicitly if it existed or general project scope understanding from phase 3. Since this file is not listed in recent context, I will infer from the general problem description and the nature of a research proposal.)}

This research focuses on the design and evaluation of an Agent Communication Protocol (ACP) for human-autonomy teaming specifically within the domain of Distributed Energy Resource (DER) management, with a particular emphasis on predictive maintenance and operational decision support. The scope includes identifying communication requirements, designing the protocol, and developing a prototype for evaluation in a simulated environment.

\textbf{Inclusions:}

\begin{itemize}
\item   Development of an ACP tailored for human-DER agent interaction.
\item   Integration of explainability (XAI) principles.
\item   Evaluation metrics focusing on team performance, trust, workload, and situational awareness.
\item   Use of a simulated DER environment for testing.
\end{enumerate}

\textbf{Exclusions:}

\begin{itemize}
\item   Full-scale implementation in a live, real-world DER system.
\item   Development of novel AI algorithms for DER control or prediction (the focus is on communication).
\item   Hardware development or integration.
\item   Cybersecurity aspects of the ACP, beyond foundational considerations.
\end{enumerate}

\textbf{Limitations:}

\begin{itemize}
\item   The evaluation will be conducted in a simulated environment, which may not fully capture the complexities of real-world operations.
\item   The participant pool for evaluation may be limited, potentially impacting the generalizability of findings.
\item   The study will focus on specific types of DERs and operational scenarios, chosen for their relevance to the research questions.
\item   The timeframe of the Master thesis imposes constraints on the depth of exploration for each objective.
\end{enumerate}

\section{4. Theoretical Framework (Target: 250 words)}

\emph{(Primary Sources: \texttt{docs/4.2.2.1-key-concepts-updated.md}, \texttt{docs/4.2.2.2-define-relationships-updated.md})}
\emph{(Supporting Sources: \texttt{docs/3.6.1-key-concepts.md})}

\subsection{4.1. Key Concepts (Approx. 120 words)}
This research is built upon several key concepts from human-computer interaction, artificial intelligence, and cognitive engineering.

\begin{itemize}
\item   \textbf{Human-Autonomy Teaming (HAT):} A collaborative paradigm where humans and autonomous systems work interdependently towards shared goals, leveraging respective strengths \cite{CitationNeeded_HATDefinition}.
\item   \textbf{Agent Communication Protocol (ACP):} A formalized set of rules and message formats governing interactions between agents (human or artificial) in a multi-agent system \cite{CitationNeeded_ACPDefinition}.
\item   \textbf{Explainable AI (XAI):} AI systems that can provide understandable justifications for their decisions and actions, crucial for building trust and enabling effective human oversight [CitationNeeded\emph{XAIDefinition, ReferTo}docs/4.2.2.1-key-concepts-updated.md].
\item   \textbf{Shared Situational Awareness (SSA):} The degree to which team members (human and AI) possess a common understanding of the current state, tasks, and goals \cite{CitationNeeded_SSADefinition}.
\item   \textbf{Distributed Energy Resources (DERs):} Smaller, decentralized power generation and storage resources connected to the grid, such as solar panels, wind turbines, and batteries \cite{CitationNeeded_DERDefinition}.
\end{enumerate}

\subsection{4.2. Conceptual Model (Approx. 130 words)}
The proposed theoretical framework posits that an ACP designed with explicit considerations for XAI, context-awareness, and trust mechanisms will significantly improve HAT effectiveness in DER management. The model (visualized in \texttt{docs/4.2.3-visual-representation.tex}) illustrates the human operator and AI agent as partners interacting through the ACP.
% Placeholder: Digital Twin Visual Representation from \texttt{docs/4.2.3-visual-representation.tex}
This protocol mediates the exchange of information, commands, and explanations. The framework suggests that by enhancing the quality of communication (clarity, timeliness, relevance, explainability), the ACP will lead to improved SSA, better decision-making, reduced operator workload, increased trust in automation, and ultimately, more resilient and efficient DER operations. The relationships between these concepts are detailed in \texttt{docs/4.2.2.2-define-relationships-updated.md}.

\section{5. Research Methodology (Target: 350 words)}

\emph{(Primary Sources: \texttt{docs/5.3.1-methodology-justification.md}, \texttt{docs/5.2.4-feasibility-analysis.md})}
\emph{(Supporting Sources: \texttt{docs/5.2.1-methodology-comparison-matrix.md}, \texttt{docs/5.3.2-methodology-limitations.md})}
\emph{(Key References: \texttt{sources/5.1.1-relevant-methodologies.json} to \texttt{sources/5.2.4-feasibility-analysis.json})}

\subsection{5.1. Selected Methodology Justification (Approx. 120 words)}
This research will adopt a mixed-methods approach, combining qualitative and quantitative research techniques, grounded in Design Science Research (DSR). DSR is suitable for creating and evaluating novel IT artifacts (in this case, the ACP) that solve organizational problems \cite{CitationNeeded_DSRHevner}. The justification for this choice is based on a comprehensive evaluation of alternatives (see \texttt{docs/5.2.1-methodology-comparison-matrix.md} and \texttt{docs/5.3.1-methodology-justification.md}), which concluded that a mixed-methods DSR approach offers the necessary rigor for artifact design, iterative development, and robust evaluation against the research objectives. The feasibility analysis (\texttt{docs/5.2.4-feasibility-analysis.md}) supports the practicality of this approach within the project constraints.

\subsection{5.2. Alternative Methodologies Considered (Approx. 80 words)}
Several alternative methodologies were considered, including purely qualitative case studies, quantitative experimental designs, and action research. While each offers strengths, they were deemed less suitable for the primary goal of artifact creation and iterative refinement. For instance, a purely quantitative experiment might not capture the nuanced user feedback essential for design, while a case study alone wouldn't provide generalizable design principles (summary from \texttt{docs/5.2.1-methodology-comparison-matrix.md} and \texttt{docs/5.2.2-methodology-strengths-limitations.md}).

\subsection{5.3. Data Collection and Analysis Approach (Approx. 100 words)}
Data collection will occur in phases:

\begin{enumerate}
\item  \textbf{Requirements Elicitation:} Literature review, expert interviews (DER operators, AI researchers), and analysis of existing systems (\texttt{docs/5.1.1-relevant-methodologies.md}).
\item  \textbf{Prototype Evaluation:} Usability testing, simulated task performance metrics (e.g., task completion time, error rates), questionnaires (e.g., NASA-TLX for workload, trust scales), and semi-structured interviews with participants.
Quantitative data will be analyzed using statistical methods (e.g., t-tests, ANOVA), while qualitative data (interview transcripts, open-ended feedback) will be analyzed using thematic analysis \cite{CitationNeeded_ThematicAnalysisBraunClarke}.
\end{enumerate}

\subsection{5.4. Workflow (Approx. 50 words)}
The research will follow a DSR cycle:

\begin{enumerate}
\item  Problem Identification & Motivation.
\item  Define Objectives for a Solution.
\item  Design & Development (of ACP and prototype).
\item  Demonstration (prototype in simulated environment).
\item  Evaluation.
\item  Communication.
% Placeholder: Workflow Diagram from \texttt{sources/8.1.3-workflow-simplified.tex}
Limitations of the chosen methodology are documented in \texttt{docs/5.3.2-methodology-limitations.md}.
\end{enumerate}

\section{6. Ethics and Sustainability (Target: 210 words)}

\emph{(Primary Sources: \texttt{docs/6.1.2-ethical-concerns-analysis.md}, \texttt{docs/6.2.1-environmental-aspects.md}, \texttt{docs/6.2.6-mitigation-strategies.md})}
\emph{(Supporting Sources: \texttt{docs/6.1.3-data-privacy-framework.md})}
\emph{(Key References: \texttt{sources/6.1.1-ethics-guidelines-review.json} to \texttt{sources/6.2.6-mitigation-strategies-detailed.json})}

\subsection{6.1. Ethical Considerations (Approx. 70 words)}
Key ethical considerations, as detailed in \texttt{docs/6.1.2-ethical-concerns-analysis.md} and \texttt{docs/6.1.1-ethics-guidelines-review.md}, include informed consent from all research participants, data anonymity and confidentiality, and ensuring that the AI system does not perpetuate biases or lead to deskilling of human operators. The data privacy framework (\texttt{docs/6.1.3-data-privacy-framework.md}) will guide data handling. Approval from the relevant ethics committee will be sought (\texttt{docs/6.1.5-approval-needs.md}).
% Placeholder: Stakeholder Context Table (ethics relevant parts) from \texttt{sources/8.1.3-stakeholder-context-table.md}

\subsection{6.2. Sustainability Dimensions and SDG Alignment (Approx. 70 words)}
This research aligns with several UN Sustainable Development Goals (SDGs), particularly SDG 7 (Affordable and Clean Energy) and SDG 9 (Industry, Innovation, and Infrastructure) by aiming to improve the efficiency and reliability of DERs (\texttt{docs/6.2.1-environmental-aspects.md}, \texttt{docs/6.2.2-social-impacts.md}, \texttt{docs/6.2.3-economic-factors.md}). Enhanced DER management can contribute to a more sustainable energy future. Equity aspects are considered in \texttt{docs/6.2.4-equity-aspects.md}, and policy implications in \texttt{docs/6.2.5-policy-implications.md}.

\subsection{6.3. Mitigation Strategies (Approx. 70 words)}
Potential negative impacts, such as over-reliance on AI or job displacement, will be explored. Mitigation strategies (\texttt{docs/6.2.6-mitigation-strategies.md}) include designing the ACP to empower human operators rather than replace them, focusing on decision support, and providing transparent AI explanations to maintain operator skills and engagement.

\section{7. Risk Assessment and Timeline (Target: 210 words)}

\emph{(Primary Sources: \texttt{docs/7.2-research-specific-risk-management.md}, \texttt{docs/7.1.5-risk-prioritized.md})}
\emph{(Supporting Sources: \texttt{docs/5.3.3-project-timeline.md})}
\emph{(Key References: \texttt{sources/7.1.1-potential-risks-detailed.json} to \texttt{sources/7.2-combined-risk-management.json})}

\subsection{7.1. Major Risk Categories and Mitigation (Approx. 100 words)}
Major risks, identified in \texttt{docs/7.1.1-potential-risks.md} and prioritized in \texttt{docs/7.1.5-risk-prioritized.md}, include:

\begin{enumerate}
\item  \textbf{Technical Risks:} Difficulties in prototype development or integration. Mitigation: Iterative development, use of established platforms.
\item  \textbf{Participant Recruitment:} Challenges in finding suitable participants for evaluation. Mitigation: Early planning, flexible scheduling.
\item  \textbf{Data Quality:} Insufficient or biased data from evaluations. Mitigation: Robust experimental design, validated instruments.
Detailed mitigation strategies are in \texttt{docs/7.2-research-specific-risk-management.md}.
% Placeholder: Risk Management Table from \texttt{sources/8.1.3-risk-management-table.md}
\end{enumerate}

\subsection{7.2. Implementation Timeline Overview (Approx. 70 words)}
The project is planned over 19 weeks, as per \texttt{docs/5.3.3-project-timeline.md}.

\begin{itemize}
\item   Weeks 1-4: Finalize literature review, ethics approval, detailed design.
\item   Weeks 5-10: ACP development and prototype implementation.
\item   Weeks 11-14: Pilot testing and participant recruitment.
\item   Weeks 15-17: Evaluation studies and data analysis.
\item   Weeks 18-19: Thesis writing and submission.
% Placeholder: Timeline Gantt Chart from \texttt{sources/8.1.3-timeline-gantt.tex}
\end{enumerate}

\subsection{7.3. Resource Requirements (Approx. 40 words)}
Required resources include access to computing facilities, simulation software, relevant academic databases, and participants for evaluation studies. These are assessed in \texttt{docs/5.2.3-resource-requirements-assessment.md}.

\section{8. Expected Results and Contributions (Target: 200 words)}

\emph{(Newly written, based on overall project understanding and aims)}

This research is expected to produce several key outcomes:

\begin{enumerate}
\item  \textbf{A Novel Agent Communication Protocol (ACP):} A well-defined and theoretically grounded ACP tailored for human-AI teaming in DER management, incorporating XAI principles.
\item  \textbf{A Prototype System:} A functional prototype demonstrating the ACP in a simulated DER environment.
\item  \textbf{Empirical Evaluation Findings:} Quantitative and qualitative data on the impact of the ACP on team performance, operator workload, trust, and situational awareness.
\item  \textbf{Design Guidelines:} A set of actionable design principles for developing effective human-agent communication systems in complex, dynamic domains.
\end{enumerate}

\textbf{Contributions:}

\begin{itemize}
\item   \textbf{Theoretical:} Advances understanding of human-autonomy teaming by proposing and evaluating a specialized ACP framework. Contributes to XAI research by applying its principles in a novel context.
\item   \textbf{Methodological:} Demonstrates the application of a mixed-methods DSR approach to a complex socio-technical problem in the energy sector.
\item   \textbf{Practical:} Offers a potential solution to improve the management of DERs, enhancing grid stability and efficiency. Provides insights for designers and developers of future human-AI systems in critical infrastructure.
The findings will be disseminated through the Master thesis and potentially conference publications or journal articles.
\end{enumerate}

\section{9. References}

\emph{(This section will be populated by BibTeX. In Markdown, list key references or use placeholders like [AuthorYear]).}
\emph{(Example: \cite{HevnerEtAl2007_DSR}) }
\emph{(All citations from \texttt{sources/8.1.4-all-citations.csv} that are used in the text will be formally listed here from the bib file.)}

\section{10. Appendices (Optional)}

\emph{(As per \texttt{docs/8.1.1-word-count-allocation.md}, this is not counted in main text. Visuals like large tables mentioned in \texttt{docs/8.1.3-visual-planning-summary.md} could go here if too large for main text.)}


\begin{itemize}
\item   \textbf{Appendix A:} Methodology Comparison Table (from \texttt{sources/8.1.3-methodology-comparison-table.md})
\item   \textbf{Appendix B:} Detailed Risk Register (condensed from \texttt{docs/7.1.3-risk-register.md} or linked to \texttt{sources/8.1.3-risk-management-table.md})
\item   \textbf{Appendix C:} Stakeholder Context Table (from \texttt{sources/8.1.3-stakeholder-context-table.md})
\end{itemize}
